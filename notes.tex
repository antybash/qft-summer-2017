%        File: notes.tex
%     Created: Tue May 17 09:00 AM 2016 E
% Last Change: Tue May 17 09:00 AM 2016 E
%
\documentclass[10pt]{article}

\usepackage{graphicx}
\usepackage{braket}
\usepackage{mdframed}
\usepackage{bm}
\usepackage{amssymb}
\usepackage{amsthm}
\usepackage{amsmath}
\usepackage{enumerate}
\usepackage{amsmath}
\usepackage{mathrsfs}
\usepackage{multicol}
\usepackage{verbatim}
\usepackage{dsfont}
\usepackage[usenames,dvipsnames,svgnames,table]{xcolor}
\usepackage[colorlinks=true,urlcolor=blue,bookmarks=true,citecolor=blue]{hyperref}
%\usepackage{color}
%\usepackage[utf8]{inputenc}
%\usepackage{fourier}
\usepackage{mathtools}
\usepackage{todo}
\usepackage{slashed}
\usepackage{tikz}
\usepackage{tikz-cd}

%\usepackage[T1]{fontenc}
%\usepackage[euler-digits,small]{eulervm}
%\usepackage[sc,osf]{mathpazo}

\oddsidemargin -0.04cm \evensidemargin -0.04cm
\setlength{\topmargin}{-0.5in} \textwidth 16.59cm \textheight 23cm 

%\usepackage{geometry}
%\geometry{
%letterpaper,
%lmargin=1cm,
%rmargin=2cm,
%tmargin=2cm,
%bmargin=2cm,
%footskip=12pt,
%headheight=12pt}

\newcommand{\iden}{\mathds{1}}

\newcommand{\bZ}{\mathbb Z}
\newcommand{\bN}{\mathbb{N}}
\newcommand{\bQ}{\mathbb{Q}}
\newcommand{\bR}{\mathbb{R}}
\newcommand{\bC}{\mathbb{C}}
\newcommand{\PH}{\mbb P\ms H}
\newcommand{\tat}{\text}
\newcommand{\ti}{\to\infty}
\newcommand{\mc}{\mathcal}
\newcommand{\ms}{\mathscr}
\newcommand{\mf}{\mathfrak}
\newcommand{\mbb}{\mathbb}
\newcommand{\cnj}{\overline}
\newcommand{\veps}{\varepsilon}
\newcommand{\sg}{\sigma}

\newcommand{\restr}{\upharpoonright}
\newcommand{\FR}[2]{\frac{#1}{#2}}
\newcommand{\PFR}[2]{\left(\frac{#1}{#2}\right)}
\newcommand{\SFR}[2]{\sqrt{\frac{#1}{#2}}}
\newcommand{\PP}[2]{\frac{\partial #1}{\partial #2}}
\newcommand{\dsum}{\oplus}
\newcommand{\ten}{\otimes}
\newcommand{\non}{\nonumber}

\newcommand{\Align}[1]{\begin{align*}#1\end{align*}}

\theoremstyle{plain}
\newtheorem{theorem}{Theorem}[section] 
\newtheorem{lemma}[theorem]{Lemma}

\theoremstyle{definition}
\newtheorem{definition}{Definition}
\newtheorem{defn}{Definition}
\newtheorem{prop}{Proposition}
\newtheorem{conjecture}{Conjecture}
\newtheorem{example}{Example}

\theoremstyle{remark}
\newtheorem*{remark}{Remark}
\newtheorem*{note}{Note}
\newtheorem{case}{Case}
\newtheorem*{claim}{Claim}

\DeclareMathOperator{\spec}{Spec}
\DeclareMathOperator{\Card}{card}
\DeclareMathOperator{\Span}{span}
\DeclareMathOperator{\rank}{rank}
\DeclareMathOperator{\real}{Re}
\DeclareMathOperator{\diam}{diam}
\DeclareMathOperator{\id}{id}
\DeclareMathOperator{\GL}{GL}
\DeclareMathOperator{\PGL}{PGL}
\DeclareMathOperator{\SL}{SL}
\DeclareMathOperator{\SO}{SO}
\DeclareMathOperator{\SU}{SU}
\DeclareMathOperator{\Det}{Det}
\DeclareMathOperator{\Tr}{Tr}
\DeclareMathOperator{\sgn}{sgn}
\DeclareMathOperator{\Map}{Map}
\DeclareMathOperator{\Stab}{Stab}
\DeclareMathOperator{\rP}{P}
\DeclareMathOperator{\rT}{T}

\def\sm{\setminus}
\def\seq{\subseteq}
\def\ii{\item}
\def\bE{\begin{enumerate}}
\def\eE{\end{enumerate}}
\def\bP{\begin{pmatrix}}
\def\eP{\end{pmatrix}}

\renewcommand{\empty}{\varnothing}
\newcommand{\hv}[1]{\hat{\textbf{#1}}}
\newcommand{\oti}[2]{#1_{#2=1}^\infty}
\newcommand{\lam}{\lambda}
\newcommand{\om}{\omega}
\newcommand{\Om}{\Omega}
\newcommand{\gam}{\gamma}
\newcommand{\di}{\partial}

\newcommand{\fs}{\slashed}

\newcommand{\ha}{\hat a}
\newcommand{\had}{\hat a^\dagger}

\newcommand{\Question}{\textbf{Question: }}
\newcommand{\Solution}{\textbf{Solution: }}
\newcommand{\const}{\textrm{const}}

\newcommand{\colr}[1]{ {\color{red}  #1 } }
\newcommand{\colb}[1]{ {\color{blue} #1 } }
\newcommand{\colnb}[1]{ {\color{NavyBlue} #1 } }
\newcommand{\colm}[1]{ {\color{Fuchsia} #1 } }

\begin{document}
\date{}
\title{}
\author{}


\section{ Relativistic Quantum Mechanics }
%

\subsection{Summary}
\begin{enumerate}
    \item Quantum Mechanics: Axioms of QM: Hilbert spaces, rays, measurements.
    \item Symmetries
        \begin{itemize}
            \item How does one relate the states that $\ms O$ measures with the states that $\ms O'$ measures?
            \item Wigner's theorem: ``For any probability-preserving transformation, is either unitary and linear, or antiunitary and antilinear.'' 
            \item Trivial symmetry, $U=\iden$. Continuity: symmetries in the path-connected component of $\iden$ are all unitary and linear. 
            \item Infinitesimally close to $\iden$: ``generators'' $\iff$ ``observables''.
            \item Representations, projective representations, superselection rules.
            \item Connected Lie groups and applications: Lie algebras and Abelian Lie groups.
        \end{itemize}
    \item Quantum Lorentz Transformations: Unitary representations of Lorentz Group.
        \begin{itemize}
            \item Define: (in)homogeneous, orthochronous. 
            \item Exercise: when working on the Lie algebra level, why is it okay to work with the homogeneous Lorentz group?
        \end{itemize}
    \item Poincar\'e algebra: the section where we do Lie algebra computations.
        \begin{itemize}
            \item Work infinitesimally away from identity, staying inside of the homogeneous Lorentz group.
            \item Compute commutators.
        \end{itemize}
    \item One Particle States: Representations of inhomogeneous Lorentz group (ie. Poincar\'e group).
        \begin{itemize}
            \item Reps of Poincar\'e $\leftrightarrow$ Reps of $\SL(2,\bC)$ %(\colr{Do we do them following Weinberg or apply structure theorems from Lie Groups course?})
            \item Orbit Method!
            \item Normalization of free particle states $N(p) = \SFR{k^0}{p^0}$.
            \item Proof Mass Positive Definite, Proof Mass Zero (Helicity).
        \end{itemize}
    \item Space Inversion and Time Reversal
        \begin{itemize}
            \item What are the operators corresponding to $\ms P,\ms T$ which act on the Hilbert space?
            \item If so, what are the properties of $\bm P, \bm{T}$?
            \item Casework: $M>0, M=0$ with $\bm P, \bm T$.
        \end{itemize}
    \item Projective Representations
        \begin{itemize}
            \item When are projective reps necessary? Answer: We can
                set the phase $\phi=\phi(T,\bar T) \to 0$ if the there
                are no \emph{central charges} in the Lie algebra and
                the group is simply connected.
            \item Define central charges, prove theorem (see appendix 2).
            \item That is: one algebraic condition, one topological
                condition. Investigate both to show that
                inhomogeneous Lorentz group it is \emph{not} simply
                connected!
            \item Discuss: intrinsic projective representations,
                superselection rules, spin.
        \end{itemize}
    \item Appendix A: Wigner's Theorem (see Dan Freed's article)
    \item Appendix B: Group Operators and Homotopy Classes (for Projective Reps)
    \item Appendix C: Inversions and Degenerate Multiplets
        \begin{itemize}
            \item ``Inversions might act in more complicated ways on
                degenerate multiplets of one-particle states''
            \item Discuss the cases for both $\bm P$ and $\bm T$.
        \end{itemize}
\end{enumerate}

\subsection{Symmetries} \label{section:symmetries}
Before getting started we need to do a digression on quantum foundations.
This will establish the need for theorems like Wigner's theorem on unitary
and antiunitary operators, projective representations, etc. \emph{Physical
states} will be restricted to rays in a separable Hilbert space, so our
configuration space is given by the projective Hilbert space: \(\mbb P
\ms H = \ms H\sm\{0\}/\sim\). One way to work with this projective Hilbert
space is to choose representatives from each ray. %% incomplete transition

Probabilities of certain measurement processes are given by amplitudes of
projections: \[\Pr[\Psi \to \Psi_n] = |\braket{\Psi_n|\Psi}|^2.\]

Two observers may measure the same physical system in different ways using
the same Hilbert space. The transformations that take one observer's ray to
the other must preserve probabilities: \[ \Pr[\ms R \to \ms R_n] = \Pr[\ms
R'\to\ms R_n'].\]
\begin{theorem}[Wigner 1930]\label{th:Wigner1930} Any probability
preserving transformation on a projective Hilbert space can be lifted to
either a unitary and linear operator or an antiunitary and conjugate-linear
operator on the original Hilbert space.
\end{theorem}
\begin{proof}(Sketch following Weinberg). Given a transformation of
projective Hilbert space \(T\colon \PH\to\PH\), we must construct an operator
\(U\colon \ms H \to\ms H\). We choose an orthonormal basis \(\Psi_n\) on
\(\mc H\) and we scale the image of this basis by phases to ensure the
result. More precisely, the key step in the proof is to define an new
basis \(\{\Upsilon_k = \FR{1}{\sqrt 2}\left( \Psi_k+\Psi_1
\right)\}_{k>1} \cup \{\Upsilon_1 = \Psi_1\}\) and choosing
\(U\Upsilon_k = \FR{1}{\sqrt 2}\left( U\Psi_k + U\Psi_1 \right).\)
This provides a constraint: if \(\Psi = \sum_{n=1}^\infty C_n\Psi_n\)
maps to \(U\Psi = \sum_{n=1}^\infty C_n' U\Psi_n\) then: \[|C_n|^2 =
|C_n'|^2,\qquad\qquad |C_k+C_1|^2 = |C_k'+C_1'|^2.\] These two
equations imply \(C_k/C_1 = C_k'/C_1'\) or \(C_k/C_1 = (C_k'/C_1')^*\)
which corresponds to unitary or antiunitary.  After choosing
\(C_1=C_1'\) or \(C_1=C_1'^*\), the proof proceeds to verify that the
operator \(U\) constructed in this way is well-defined,
unitary/antiunitary, and linear/conjugate-linear.
\end{proof}

\subsubsection{Representations}
Let \(G\) be a group of symmetries of our physical system. Mathematically,
we ask that this is a probability-preserving group action on $\PH$.
In particular, any element \(g\in G\) induces a transformation on the
projective Hilbert space which, using Wigner's theorem~\ref{th:Wigner1930},
corresponds to an operator on the Hilbert space \(U(g)\colon \ms H \to \ms
H.\) However, this assignment \(U\colon G \to \GL(\ms H)\) is not necessarily
a unitary or antiunitary representation; but rather a \textbf{projective
representation}. Indeed, fix a norm-one basis element \(\Psi_n\) and
consider the action of \(U_1U_2 \coloneqq U(g_1)U(g_2)\) and
\(U_{12}\coloneqq U(g_1g_2)\). Since the operators \(U_1,U_2,U_{12}\)
by construction descend to operators on $\PH$, this means that
\[U_1U_2\Psi_n = e^{i\phi_n(T_1,T_2)}U_{12}\Psi_n.\]

There is a subtlety involved in this: does the phase depend on the
choice of state $\Psi_n$? If we are able to prepare
superpositions of states \(\Psi + \Phi\) for any \(\Psi\) and \(\Phi\) then
it's easy to show that the phase in the projective representation is
independent of the basis element \(\Psi_n\). However, the ability to
construct superpositions may be prohibited by the choice of group that
we pick. This restriction is what is known as \textbf{superselection
rules}~\cite{WWW52}. If it turns out that we are indeed able to make a
certain linear combination that is not allowed by superselection
rules, this simply means that we need to slightly enlarge our group.
See the following section on projective representations for more
details.

If the group is Abelian, which is the case for space-time translations or
rotations around a fixed axis, then every unitary transformation is given
by the exponential: \[U(T(\theta)) = U(T(\theta/N))^N = \lim_{N\to\infty}
\left[ 1+\FR{it_a\theta^a}{N} \right]^N = e^{it_a\theta^a}.\]
 

\subsection{Projective Representations (Weinberg Vol I \S 2.7)}

\begin{comment}
    \subsubsection{Mathematical interlude}
    Projective representations may be thought of as maps $\rho:G\to
    \PGL(V)$ where $\PGL(V) = \GL(V)/\mbb C$. Since we still are
    interested in operators on $V$, our Hilbert space, the question
    becomes: ``Can we lift $\rho:G\to\PGL(V)$ to a representation
    $\tilde\rho:G\to \GL(V)$?'' There is a very good answer to this
    question and it involves two gadgets: central extensions and group
    cohomology.

    \begin{itemize}
        \item ``Can we lift $\rho:G\to\PGL(V)$ to a representation $\tilde\rho:G\to \GL(V)$?'' 

    In general, no. 

    \item ``What is the obstruction?''

    The gadget $H^2(G,V)$\footnote{Perhaps we should think of this as an
    analogue of differential 2-forms $\alpha$ for which $d\alpha\ne0$.} is
    a vector space that determines whether or not there are non-trivial
    projective representations of $G$.  

    \item ``Anything else interesting?''
        \begin{itemize}
            \item Central extensions classified by $H^2(G,V)$.
            \item Projective representations $\leftrightarrow$ central
                extensions.
            \item If it's not possible to lift $G\to\PGL(V)$ to
                $G\to\GL(V)$, it is possible to lift to $N\rtimes G
                \to\GL(V)$.
        \end{itemize}

    \end{itemize}



    \pagebreak 
\end{comment}

\begin{comment}

\subsubsection{Weinberg's Discussion}

Recall that all of our representations will be unitary.
\begin{defn}
    A \textbf{projective representation} of a group $G$ is a map
    $\rho : G \to U(V)$ such that
    \begin{align}
        \rho(T_1T_2) = e^{i\phi(T_1,T_2)} \rho(T_1) \rho(T_2). 
        \label{eqn:projrep}
    \end{align} 
\end{defn}
Imposing associativity, $\rho(T_1)(\ \rho(T_2)\rho(T_3)\ ) = (\
\rho(T_1)\rho(T_2)\ )\rho(T_3)$ gives a necessary condition for the
phase $\phi$ to satisfy.
\begin{align*}
    \rho(T_1T_2T_3)
    &= e^{i\phi(T_1,T_2T_3)}\rho(T_1)\rho(T_2T_3)
     = e^{i\phi(T_1,T_2T_3)+i\phi(T_2,T_3)}\rho(T_1)\rho(T_2)\rho(T_3)\\
    &= e^{i\phi(T_1T_2,T_3)}\rho(T_1T_2)\rho(T_3)
     = e^{i\phi(T_1T_2,T_3)+i\phi(T_1,T_2)}\rho(T_1)\rho(T_2)\rho(T_3).
\end{align*}
This gives the condition:
\begin{align}
    \phi(T_1,T_2T_3) + \phi(T_2,T_3) \equiv \phi(T_1T_2,T_3)+\phi(T_1,T_2) \pmod{2\pi}
    \label{eqn:phasecocycle}
\end{align}
\begin{example}[Trivial phase] \label{example:trivialphase}
    If the phase $\alpha(T_1,T_2)$ is defined by $\alpha(T_1,T_2) =
    f(T_1T_2) - f(T_1) - f(T_2)$, then this $\alpha$ satisfies
    \eqref{eqn:phasecocycle}. This is called a trivial phase because,
    by multiplying the projective representation $\rho(T)$ by
    $e^{i\alpha(T)}$ we obtain linear representation $\tilde\rho =
    \rho\cdot e^{i\alpha}$, such that $\tilde\rho(T_1T_2) =
    \tilde\rho(T_1)\tilde\rho(T_2)$.
\end{example}
\begin{defn}
    Denote by $\Map(G\times G,\bR/2\pi\bZ)$ the set of all phases
    $\phi:(T_1,T_2)\mapsto [0,2\pi]$. 
    A \textbf{2-cocycle} is an element of the group $\Map(G\times
    G,\bR/2\pi\bZ)/\sim$ where $\phi \sim \psi$ if $\phi-\psi$
    has the form of Example~\ref{example:trivialphase}.
    The trivial 2-cocycle is given by the equivalence class containing
    the $\phi\equiv0$ phase. 
\end{defn}

\begin{remark}
    Consider a finite dimensional connected Lie group $G$, with $\dim
    G = d$. Choose coordinates $\{\theta^a\}_{a=1}^d$ centred at the
    identity of $G$, or more precisely, $f:U\seq\bR^d\to G$ will be the
    chart near the identity for some open set $U$ containing $0$.
    %
    Assume that $\phi\in\Map(G\times G,\bR/2\pi\bZ)$ is at least two
    times continuously differentiable at $0$ and let's consider what
    happens when we perturb $\phi$ away from the point
    $(\iden,\iden)\in G\times G$.
    %%%

    With this parameterization, we can pull back $\phi:G\times G\to
    \bR/2\pi\bZ$ to a map $f^*\phi:\bR^d\times \bR^d \to \bR/2\pi\bZ$:
    \[ f^*\phi(\theta,\bar\theta) \coloneqq \phi(\ T(\theta),T(\bar\theta)\ ). \]
    Since $f^*\phi(\theta,0) = f^*\phi(0,\bar\theta) = \iden$, it
    follows that $\di_\theta(f^*\phi)=\di_{\bar\theta}(f^*\phi)=0$,
    moreover $f^*\phi(0,0) = \phi(\iden,\iden) = 0$ which means that
    the Taylor expansion starts at least at second order:
    \[ \phi(\ f(\theta),f(\bar\theta)\ ) = f^*\phi(\theta,\bar\theta)
    = \phi_{ab}\theta^a\bar\theta^b+\cdots\qquad\text{where}\; \phi_{ab}\in\bR.\]

    
    %
    Let us substitute this Taylor series into equation~\eqref{eqn:projrep}:
    \begin{align*}
        \rho(\ f(\theta)f(\bar\theta)\ ) &= e^{if^*\phi(\theta,\bar\theta)} \rho(f(\theta))\,\rho(f(\bar\theta))\\
        LHS 
        &= \rho(\ (\iden+(\di_a f)t_a+\cdots)(\iden+(\di_b f)t_b+\cdots)\ )\\
        &= \rho(\ (\iden+(\di_a f)t_a+\cdots)(\iden+(\di_b f)t_b+\cdots)\ )\\
    \end{align*}

\end{remark}

\pagebreak 
\end{comment}

\subsubsection{Introduction}

Recall that all of our representations will be unitary.
\begin{defn}
    A \textbf{projective representation} of a group $G$ is a map
    $\rho : G \to U(V)$ such that
    \begin{align}
        \rho(T_1T_2) = e^{i\phi(T_1,T_2)} \rho(T_1) \rho(T_2). 
        \label{eqn:projrep1}
    \end{align} 
\end{defn}
Imposing associativity, $\rho(T_1)(\ \rho(T_2)\rho(T_3)\ ) = (\
\rho(T_1)\rho(T_2)\ )\rho(T_3)$ gives a necessary condition for the
phase $\phi$ to satisfy.
\begin{align*}
    \rho(T_1T_2T_3)
    &= e^{i\phi(T_1,T_2T_3)}\rho(T_1)\rho(T_2T_3)
     = e^{i\phi(T_1,T_2T_3)+i\phi(T_2,T_3)}\rho(T_1)\rho(T_2)\rho(T_3)\\
    &= e^{i\phi(T_1T_2,T_3)}\rho(T_1T_2)\rho(T_3)
     = e^{i\phi(T_1T_2,T_3)+i\phi(T_1,T_2)}\rho(T_1)\rho(T_2)\rho(T_3).
\end{align*}
This gives the condition:
\begin{align}
    \phi(T_1,T_2T_3) + \phi(T_2,T_3) \equiv
    \phi(T_1T_2,T_3)+\phi(T_1,T_2) \pmod{2\pi}
    \label{eqn:phasecocycle1}
\end{align}

The following example and definition is to make precise the meaning of
equivalent representations.
\begin{example}[Trivial phase] \label{example:trivialphase1}
    If the phase $\alpha(T_1,T_2)$ is defined by $\alpha(T_1,T_2) =
    f(T_1T_2) - f(T_1) - f(T_2)$, then this $\alpha$ satisfies
    \eqref{eqn:phasecocycle1}. This is called a trivial phase because,
    by multiplying the projective representation $\rho(T)$ by
    $e^{i\alpha(T)}$ we obtain linear representation $\tilde\rho =
    \rho\cdot e^{i\alpha}$, such that $\tilde\rho(T_1T_2) =
    \tilde\rho(T_1)\tilde\rho(T_2)$.
\end{example}
%
\begin{defn}
    Denote by $\Map(G\times G,\bR/2\pi\bZ)$ the set of all phases
    $\phi:(T_1,T_2)\mapsto [0,2\pi]$. 
    A \textbf{2-cocycle} is an element of the group $\Map(G\times
    G,\bR/2\pi\bZ)/\sim$ where $\phi \sim \psi$ if $\phi-\psi$
    has the form of Example~\ref{example:trivialphase1}.
    The trivial 2-cocycle is given by the equivalence class containing
    the $\phi\equiv0$ phase. 
\end{defn}

\begin{remark}
    At this point we shall make a deal with the devil, but only once
    in this section: let us identify $\mf g$ with $G$. This can be
    done in a neighbourhood of the identity using the exponential
    map, but our deal will consist applying Taylor's formula
    na\"ively. We do this only for clarity's sake.
\end{remark}

Let $\vec\theta = (\theta^1,\dots) \equiv \{\theta^a\}$ parameterize
the group elements $T(\theta)$ near the identity. We must express the
following statements on the level of the Lie algebra: group
multiplication, representation, phase.
%
Applying Taylor's formula, we find constants $f^a_{\ bc}$,
%
\begin{align}
\rho (T(\theta))        &= \iden + i\theta^a\mf t_a + \FR{1}{2}\theta^b\theta^c\mf t_{bc} + \cdots\\
f^a(\theta,\bar\theta)  &= \theta^a + \bar\theta^a + f^a_{\ bc}\theta^b\bar\theta^c + \cdots\\
\phi(\theta,\bar\theta) &= \phi_{bc}\theta^b\bar\theta^c + \cdots
\label{}
\end{align}

Let's substitute these three equations to get the infinitesimal
equivalent of $\rho(T_1)\rho(T_2) = e^{i\phi(T_1,T_2)}\rho(T_1T_2)$:
\begin{align}
\rho(T(\theta))\rho(T(\bar\theta)) &=
e^{i\phi(T(\theta),T(\bar\theta))}\rho(T_{f(\theta,\bar\theta)})\\
%
(\iden + i\theta^a\mf t_a + \FR{1}{2}\theta^b\theta^c\mf t_{bc})
(\iden + i\bar\theta^a\mf t_a + \FR{1}{2}\bar\theta^b\bar\theta^c\mf t_{bc})
&=
(1 + i\phi_{bc}\theta^b\bar\theta^c)
(\iden + if^a\mf t_a + \FR{1}{2}f^b f^c\mf t_{bc})\\
&=
(1 + i\phi_{bc}\theta^b\bar\theta^c)
\biggl[\iden + i(\theta^a+\bar\theta^a+f^{a}_{\ bc}\theta^b\bar\theta^c)\mf t_a\\
&\qquad\qquad\qquad\qquad
+\FR{1}{2}(\theta^b+\bar\theta^b) (\theta^c+\bar\theta^c)\mf t_{bc}\biggr]
\end{align}
The $\theta,\bar\theta,\theta\theta,\bar\theta\bar\theta$ terms are
equal on both sides, up to second order. Let's look at the
$\theta\bar\theta$ terms on both sides:
\begin{align}
    LHS_{\theta\bar\theta} &= (-\mf t_b\mf t_c)\theta^b\bar\theta^c\\
    RHS_{\theta\bar\theta} &= (if^{a}_{\ bc}\mf t_a + \mf t_{bc} + i\phi_{bc}\iden )\theta^b\bar\theta^c\\
    \implies\qquad
    \mf t_{bc} &= -\mf t_b\mf t_c - if^{a}_{\ bc} - i\phi_{bc}\iden
\end{align}
By our choice of $\mf t_{bc}$ being symmetric with respect to $b,c$,
we have an additional relation $\mf t_{bc} = \mf t_{cb}$. Rearranging
we easily get:
\begin{align}
    [\mf t_b,\mf t_c] &= -i(f^a_{\ bc} - f^a_{\ cb})\mf t_a + i(\phi_{bc}-\phi_cb)\\
    &= iC^{a}_{\ bc}\mf t^a + iC_{bc}\iden
\end{align}
\begin{defn} When we have a linear representation, ie. $\phi(T_1,T_2)
    = 0$ for all $T_1,T_2$, then $C^{a}_{\ bc}$ completely define the
    Lie algebra and are called the \textbf{structure constants}.
    When $\phi$ is non-zero, then the $C_{bc}$'s are called the
    \textbf{central charges}.
\end{defn}

\subsubsection{Sufficient Conditions: Turning 
Projective Representations into Affine Representations}

The key relation that we have come to is $[\mf t_b,\mf t_c] =
iC^{a}_{\ bc}\mf t^a + iC_{bc}\iden$. A necessary condition of the
generators of a Lie algebra to satisfy is the Jacobi identity. These
will lead to a pair of equations. Combining this with a topological
condition we shall arrive at a theorem which gives sufficient
conditions which allow making projective representations into linear
representations.

\newcommand{\mt}{\mf t}
Let us denote $cyc$ to be the cyclic permutation of $(a,b,c)$ in that order.
The Jacobi identity reads:
%   \begin{align*}
%   [\mt_a,[\mt_b,\mt_c]]+ [\mt_b,[\mt_c,\mt_a]]+ [\mt_c,[\mt_a,\mt_b]]&=0\\
%   [\mt_a,iC^d_{\ bc}\mt_d + i C_{bc}\iden]+
%   [\mt_b,iC^d_{\ ca}\mt_d + i C_{ca}\iden]+
%   [\mt_c,iC^d_{\ ab}\mt_d + i C_{ab}\iden]&=0\\
%       iC^d_{\ bc}[\mt_a,\mt_d]+ i C_{bc}[\mt_a,\iden]+
%       iC^d_{\ ca}[\mt_b,\mt_d]+ i C_{ca}[\mt_b,\iden]+
%       iC^d_{\ ab}[\mt_c,\mt_d]+ i C_{ab}[\mt_c,\iden]&=0\\
%   iC^d_{\ bc}[\mt_a,\mt_d]+
%   iC^d_{\ ca}[\mt_b,\mt_d]+
%   iC^d_{\ ab}[\mt_c,\mt_d]&=0\\
%       iC^e_{\ bc}(iC^e_{\ ad}t_e+ i C_{ad}\iden)+
%       iC^e_{\ ca}(iC^e_{\ bd}t_e+ i C_{bd}\iden)+
%       iC^e_{\ ab}(iC^e_{\ cd}t_e+ i C_{cd}\iden)&=0\\
%   \end{align*}
\begin{align*}
[\mt_a,[\mt_b,\mt_c]]+ cyc &= 0\\
[\mt_a,iC^d_{\ bc}\mt_d + i C_{bc}\iden]+ cyc &= 0\\
    iC^d_{\ bc}[\mt_a,\mt_d]+ i C_{bc}[\mt_a,\iden]+ cyc &= 0\\
iC^d_{\ bc}[\mt_a,\mt_d] + cyc &=0\\
    iC^d_{\ bc}(iC^e_{\ ad}\mt_e+ i C_{ad}\iden)+ cyc &=0
\end{align*}
The coefficient of $\mf t_e$ and $\iden$ must be zero, which gives the
following two relations:
\begin{align*}
    0 &= C^d_{\ bc}C^e_{\ ad} + cyc = 
    C^d_{\ bc}C^e_{\ ad} + C^d_{\ ca}C^e_{\ bd} + C^d_{\ ab}C^e_{\ cd}\\
    0 &= C^d_{\ bc}C_{\ ad} + cyc = 
    C^d_{\ bc}C_{ad} + C^d_{\ ca}C_{bd} + C^d_{\ ab}C_{cd}
\end{align*}
There is an immediate way to reduce the second equation to the first:
set $C_{ab} = C^{e}_{\ ab}\phi_e$ for some choice of $\phi_e$. In
particular, shifting $\mt_a\to\tilde\mt_a = \mt_a + \phi_a\iden$ we get:
\begin{align*}
    [\tilde\mt_a,\tilde\mt_b]
&= [\mt_b + \phi_b\iden,\mt_c+\phi_c\iden] = [\mt_b,\mt_c]
%&= iC^{a}_{\ bc}\mt_a + iC_{bc}\iden\\
= iC^{a}_{\ bc}\mt_a + iC^{a}_{bc}\phi_a\iden
= iC^{a}_{\ bc}\tilde\mt_a
\end{align*}

\begin{theorem}
    Let $\rho:G\to U(V)$ be a projective representation with $\dim G <
    \infty$. If there is a choice of generators $\{\mt_a\}\seq \mf
    u(V)$ that may be extended to a Lie-algebra basis on $\mf u(V)$,
    that is the $\{\mt_a\}$ do not give rise to central charges, and
    if $G$ is simply connected, then we may ``choose phases of the
    operators to obtain a linear representation.''
\end{theorem}

\renewcommand{\th}{\theta}
\begin{proof}{\ }
    Weinberg's proof is clear. Here is an outline of his exposition.
    What do we want? A linear representation. How will we get one? By
    brute force: writing down a differential equation that the
    `representation operators' must satisfy, finding the properties of
    such operators, show that they form a linear representation. It's
    quite fun actually.
    \begin{comment}
    \begin{itemize}
        \item Recall: construction of $\rho(T)$
    \begin{itemize}
        \item Choose $\theta$,$f$ so that $T(\th)T(\th')= T(f(\th,\th'))$
        \item 
    \end{itemize}
    \end{itemize}
    \end{comment}
\end{proof}

\begin{mdframed}
    Big Question: If we are given a projective transformation of the
    inhomogeneous orthochronous Lorentz group, can we redefine the
    operators to make a \emph{linear} representation?  In other words,
    can we apply the above theorem to the Lorentz group?
\end{mdframed}

We need to check the algebraic hypothesis and the topological
hypothesis: it will turn out that the topological condition fails.

\begin{enumerate}
    \item (Topological Condition)
        \begin{enumerate}
            \item $\bR^{3,1} \xrightarrow{\;\cong\;} \mf u(2,\bC)$
                \[ V^\mu \mapsto v = V^\mu\sg_\mu = \bP V^0+V^3 & V^1
                - iV^2 \\ V^1+iV^2 & V^0 - V^3 \eP\]
            \item Let $\lam\in\mf{gl}(2,\bC)$, with $|\det\lam|=1$.
                \begin{align*}
                    v &\to \lam v \lam^\dag   & 
                    V^\mu\sg_\mu \to 
                    \lam V^\mu\sg_\mu \lam^\dag = 
                    (\Lambda^\mu_\nu(\lam) V^\nu)\sg_\mu
                \end{align*}
                where there is a unique $\Lambda(\lam)$ since $V^\mu
                V_\mu = -\det v$ and since conjugation by $\lam$
                (having $|\det\lam|=1$) perserves the inner product.
            \item Property: $\Lambda(\lam)\Lambda(\lam') = \Lambda(\lam\lam')$.
            \item Property: If $\lam' = e^{i\theta}\lam$, then
                conjugation by $\lam$ and $\lam'$ is equivalent.
                Therefore we have an induced action of $\SL_2(\bC)$ on
                $\bR^{3,1}$.
            \item Property: if $\lam\in\SL_2(\bC)$, then
                $-\lam\in\SL_2(\bC)$ and produces the same Lorentz
                transformation.
                \begin{prop}
                    $\ms L \cong \SL_2(\bC)/\bZ_2$, where $\ms L$ is
                    the homogeneous Lorentz group.
                \end{prop}
            \item Topology of $\SL_2(\bC)$: Polar Decomposition.
                For any $\lam\in\SL_2(\bC)$, there exist unique $u\in
                U(2)$, $h\in i\mf u(2) = \mathrm{Hermitian}(2)$, such
                that $\lam = u e^{h}$. Exercise: since $\lam \in
                \SL_2(\bC)$ show that: 
                \[ u\in \SU(2), \qquad \Tr h = 0. \]
                Traceless, Hermitian matrices can be written uniquely
                as $n\cdot\vec \sg$ where $\vec\sg$ are the Pauli
                matrices; $\SU(2)\cong S^3$, the 3-sphere.  Therefore:
                \begin{align*}
                \SL_2(\bC)&\xrightarrow{\;homeo.\;}\bR^3\times S^3\\
\text{homogeneous Lorentz group}&\xrightarrow{\;homeo.\;}
    \bR^3\times (S^3/\bZ^2) \cong \bR^3\times \bR\rm P^2\\
\text{inhomogeneous Lorentz group}&\xrightarrow{\;homeo.\;}
   \bR^4 \times \bR^3\times (S^3/\bZ^2) \cong \bR^7\times \bR\rm P^2\\
                \end{align*}
        \end{enumerate}

    \item (Algebraic Condition)

        %%%%
        Fun fact (only applicable to subgroup of inhom. Lorentz
        group, the $\{J^{\mu\nu}\}'s$):
        \begin{prop}
            A semi-simple Lie algebra allows no central charges.
            (Remember: central charges are properties of the Lie
            algebra of the target space of a representation.)
        \end{prop}
        This proposition does \textbf{\emph{NOT}} apply to the entire
        Lorentz group.
\end{enumerate}

Combining the topological description of the inhomogeneous Lorentz
group and the proof of the theorem we may conclude that the phase can
\emph{almost} be eliminated. In fact, since the fundamental
group, $\pi_1(\text{inhom.  Lorentz group}) = \bZ_2$, we may conclude
that ``because the double loop that goes twice from $1$ to $\Lambda$
to $\Lambda\bar\Lambda$ and then back to $1$ \emph{can} be contracted
to a point, so we must have \[\left[ U(\Lambda)U(\bar\Lambda)
U^{-1}(\Lambda\bar\Lambda)\right]^2 = \iden,\] and hence the phase
$e^{i\phi(\Lambda,\bar\Lambda)} = \pm1$.  Similarly for the
inhomogeneous group we get the relation: 
\begin{align}
    U(\Lambda,a)U(\bar\Lambda,\bar a) = \pm
    U(\Lambda\bar\Lambda,\Lambda a+\bar a)."
    \label{eqn:superselectionrestriction}
\end{align}

\begin{defn}
    \footnote{Check this: perhaps this interpretation is not correct.}
    Fix a unitary (possibly projective) representation $U:\ms
    L \to U(V)$. If $U(\Lambda_1)U(\Lambda_2) =
    U(\Lambda_1\Lambda_2)$ for all $\Lambda_i\in\ms L$ then the set of
    vectors $v\in V$ are called integer-spin states. Otherwise, if
    there exist $\Lambda_1,\Lambda_2$ for which $U(\Lambda_1)
    U(\Lambda_2) = -U(\Lambda_1\Lambda_2)$, then the $v\in V$ are
    called a half-integer states.
\end{defn}

\begin{remark}
    In fact, in the half-integer case, we know exactly which
    $\Lambda_1,\Lambda_2$ are problematic: they are precisely the ones
    for which the path $1\to\Lambda_1\to\Lambda_1\Lambda_2\to1$ is not
    contractible to a point. 
    %
    For concreteness, suppose you have a \emph{massless} state with
    angular momentum (helicity) $\sg$ in the $z$-direction. After
    applying a rotation about the $z$-axis, the state will pick up a
    phase of $e^{2\pi i\sg}$ (check). This distinguishes integer and
    \mbox{half-integer $\sg$.}
\end{remark}
\begin{remark}
    The restriction \eqref{eqn:superselectionrestriction} imposes the
    superselection rule: ``we cannot mix integer and half-integer
    states.'' Let's spell this out in full: suppose $\psi_0$ is an
    integer-state whereas $\psi_{1/2}$ is half-integer-state.
    \[ \pm U_{12}(\psi_0 + \psi_{1/2}) = U_1U_2(\psi_0 + \psi_{1/2}) 
    = U_1U_2\psi_0 + U_1U_2\psi_{1/2} = U_{12}\psi_0 - U_{12}\psi_{1/2}
    = U_{12}(\psi_0-\psi_{1/2})\]
    Applying $U_{12}^{-1}$ to both sides we obtain from the leftmost
    equal to the rightmost expression: $\pm(\psi_0+\psi_{1/2}) =
    \psi_0-\psi_{1/2}$. This means either $\psi_{1/2}=0$ or
    $\psi_0=0$, both of which contradict the fact that these should
    represent physical states.
\end{remark}

\subsubsection{Covering spaces}
We have seen that $\SL_2(\bC) \to \ms L$ is a 2-1 covering map.
\begin{mdframed}[%
linewidth=2,roundcorner=10pt,leftmargin=20,
rightmargin=20,backgroundcolor= yellow!40,
linecolor=blue!70!black,frametitle=Question: What should we call the
Lorentz group?] 
    \begin{enumerate}
            \vspace{-.2cm}
        \item  Can we be bold and \emph{replace} what we call the
            Lorentz group ($\SL_2(\bC)/\bZ_2$) with $\SL_2(\bC)$?
            \vspace{-.2cm}
        \item If we do this, what do we gain and what do we lose?
    \end{enumerate}
\end{mdframed}

\noindent Our conventional Lorentz group, $\SL_2(\bC)/\bZ_2 \cong
\SO^+(3,1)$, has two features relevant for this discussion:
\begin{enumerate}
    \item $\SO^+(3,1)$ admits ``intrinsic'' projective representations.
    \item Topological structure implies there are two superselection sectors.
\end{enumerate}
The group $\SL_2(\bC)$, however, does not admit projective
representations because it is simply connected and semi-simple. The
nonexistence of projective representations means that $\SL_2(\bC)$ is
not able to impose superselection rules. However, the representations
(apart from being all non-projective) of $\SL_2(\bC)$ are equivalent
to the original $\SL_2(\bC)/\bZ_2$: which means that the
\emph{physical consequences} are the same modulo superselection.

To summarize, the only caveat of saying that the Lorentz group is
$\SL_2(\bC)$ is that we cannot conclude whether or not ``we can prepare
physical systems in linear combinations of states of integer and
half-integer spin, but only that the observed Lorentz invariance
($\SL_2(\bC)$) of nature cannot be used to show that such
superpositions are impossible.

This procedure applies to any Lie group $G$ that admits central
charges or is not simply connected. If it has central charges, enlarge
the group;\footnote{this has not yet been shown in these notes} if it
is not simply connected, replace $G$ with the covering space. However,
when we do this replacement we must always remember: we will not be
able to conclude whether or not all linear combinations are allowed.

To quote Weinberg:
``In short, the issue of superselection rules is a bit of a red
herring; \emph{it may or may not be possible to prepare physical
systems in arbitrary superpositions of states, but one cannot settle
the question by reference to symmetry principles, because whatever one
thinks the symmetry group of nature may be, there is always another
group whose consequences are identical except for the absence of
superselection rules.}


\subsection{Lorentz Transformations}\label{SS:rqm-lorentz}
In this subsection the relativistic part gets incorporated into our quantum
mechanical framework. The important equations are highlighted but most
of the discussion surrounding them is suppressed.

The equation that preserves proper time is given by:
\begin{align} 
\eta_{\mu\nu}dx^\mu dx^\nu = \eta_{\alpha\beta}dx'^\alpha dx'^\beta.
\label{eqn:proper-time}
\end{align}
Lorentz transformations are a subgroup of the general linear group
\(\GL(\bR^4)\) preserving proper time. Rewriting \eqref{eqn:proper-time} as
\(\eta_{\alpha\beta}\PP{x'^\alpha}{x^\mu} \PP{x'^\beta}{x^\nu} =
\eta_{\mu\nu}\), we get that:
\[ \eta_{\alpha\beta}\Lambda^\alpha_{\ \ \mu}\Lambda^\beta_{\ \ \nu} =
\eta_{\mu\nu}.\]
Taking the determinant we get \((\Det\Lambda)^2=1\). Another restriction
comes from considering the \(\Lambda^0_0\) component: \((\Lambda^0_0)^2 = 1
+ \sum_{i=1}^3\Lambda^0_i\Lambda^0_i\) which means \(\Lambda^0_0 \ge 1\) or
\(\Lambda^0_0 \le -1\). This analysis shows there are at least \textbf{four
connected components} of the Lorentz group. The $\Det\Lambda=1,
\Lambda^0_0\ge 1$ is called the proper orthochronous Lorentz group.  
\begin{theorem} Let $\ms P$ denote a parity transformation: $\ms P^0_{\ \
0} = 1, \ms P^i_{\ \ i} = -1$ for $i=1,2,3$ and let $\ms T$ denote a
time-reversal: $\ms T^0_{\ \ 0} = -1$ and $\ms T^i_{\ \ i} = 1$ for
$i=1,2,3$. Then the Lorentz group is generated by the proper, orthochronous
Lorentz group and $\ms P, \ms T, \ms P\circ\ms T$.
\end{theorem}
\subsubsection{Poincar\'e Algebra}
The \textbf{Poincar\'e group} is an extension of the Lorentz group to
include translations: $x'^\mu = \Lambda^\mu_{\ \ \nu} x^\nu + a^\mu$. To
derive the Lie algebra associated with this group we will consider an
infinitesimal transformation: \[\Lambda^\mu_{\ \ \nu} = \delta^\mu_\nu +
\omega^\mu_{\ \ \nu},\qquad\qquad a^\mu = \epsilon^\mu,\] with $\omega,
\epsilon$ infinitesimal parameters. Imposing that this infinitesimal
transformation preserves proper-time, we obtain:
\begin{align*}
    \eta_{\rho\sg} &= \eta_{\mu\nu}(\delta^\mu_{\ \
    \rho}+\om^\mu_{\ \ \rho})(\delta^\nu_{\ \ \sg}+\om^\nu_{\ \ \sg})\\
    &= \eta_{\sg\rho} + \om_{\sg\rho}+\om_{\rho\sg} + O(\om^2)
\end{align*}
Which means that $\om$ is an antisymmetric 2-tensor in 4 dimensions.
The Poincar\'e group is a symmetry of space-time and so if we suppose our
quantum theory is also invariant under this group then by the
discussion at the beginning of 
section \ref{section:symmetries} we must have a unitary operator corresponding to this infinitesimal
transformation: \[ U(\iden+\omega,\epsilon) = \iden + \FR{1}{2}i\om_{\rho\sg}
J^{\rho\sg} - i\epsilon_\mu P^\mu.\] Unitarity implies that each component
of $J$ and $P$ are Hermitian, and antisymmetry of $\om_{\rho\sg}$ implies
antisymmetry of $J^{\rho\sg}$. The sign in front of $i\epsilon_\mu P^\mu$
is a convention. %Page 60


\subsection{Poincar\'e Algebra}
The transformation properties of $J^{\mu\nu}$ and $P^\mu$ are obtained by
considering the equation:
\[ U(\Lambda,a)U(1+\om,\epsilon)U(\Lambda,a)^{-1} = U(\Lambda', a').\]
Applying a taylor expansion to both the LHS and RHS we should arrive at a
transformation law for each $J^{\mu\nu}$ and $P^\mu$. Doing this also
shows that $\mu,\nu$ are tensor indices.

The commutation relations follow from taking $\Lambda =\iden+\om$.

\subsubsection{Low-Velocity Limit}

\subsection{One-Particle States}
The Poincar\'e algebra has the following commutation relation:
$[P^\mu,P^\nu]=0$.  This, in particular, shows that
$[H,P^\mu]=[P^0,P^\mu]=0$ giving us good quantum numbers $p^\mu$ so we
shall label our basis states as:  \[ \ket{p,\sg}. \] A priori we don't
know  if $\sg$ is a discrete index or a continuous one, however we
take \emph{as part of the definition of a one-particle state} that
$\sg$ is a discrete index.

However, to define a one-particle state let's not use $\ket{p\sg}$,
ie.  just a single vector inside some Hilbert space. Let us remember
that \emph{the state of the particle depends on the observer measuring
the particle}! Therefore we shall identify the vectors \[\ket{p\sg}
\sim U(\Lambda)\ket{p\sg} = \sum_{\sg'} C_{\sg'\sg} (\Lambda)
\ket{p\sg'},\] where the notation $\ket\alpha\sim\ket\beta$ means ``we 
identify $\ket\alpha$ with $\ket\beta$.'' This motivates:
\begin{defn} A \textbf{one-particle state} is a finite-dimensional
    irreducible representation of the Lorentz group.
\end{defn}

\subsection{Classification of Finite Dimensional Irreducible
Representations of the Lorentz Group}

Our goal in this section will be to \emph{build} representations of
the homogeneous orthochronous Lorentz group which we shall denote $\ms
L$. We will do this explicitly and diligently. A rough outline of the
procedure is as follows. First, we notice there are a lot of possible
basis vectors for the Hilbert space $\ms H$. In fact, for each $p\in
\bR^{3,1}$ there are many \emph{basis} vectors $\ket{p\sg} \in \ms H$.
Second, to write down a representation we must describe the action of
each $\Lambda\in\ms L$, or more precisely $U(\Lambda)\in U(\ms H)$, on
each $\ket{p\sg}$. To accomplish this mighty task we will be even
mightier: first, we will choose only a small subset of the basis
vectors and describe the action of $\ms L$; second, we'll extend our
work to all other basis vectors.

\textbf{Part One: Simplification}
Let's push ourselves to the inside of the light-cone and restrict to
$p^2 \le 0$. Here, there are two invariant functions for $\ms L$, the
\emph{homogeneous, orthochronous} group (draw a picture of the
light-cone to convince yourself):
\begin{align*}
    f(p) &= p^2 = \eta_{\mu\nu}p^\mu p^\nu &
    g(p) &= \sgn(p^0)
\end{align*}
The pictorial proof makes one more result apparent: the action of $\ms
L$ partitions the inner-light-cone into orbits. Each orbit can be
classified/described physically. There are three relevant parts that
are of interest to us, two of which are non-trivial:
\begin{itemize}
    \item Massive orbit: $\ms L\cdot (0,0,0,M), M>0$
    \item Massless orbit: $\ms L\cdot (0,0,\kappa,\kappa),\kappa>0$
    \item Vacuum: $\ms L\cdot (0,0,0,0)$
\end{itemize}
The notation $\ms L \cdot p$ denotes the set $\{\Lambda p\}_{\Lambda
\in\ms L}$, or in words, the orbit of $p$ under the action of $\ms L$.

For each orbit, we may define a representative $k\in\bR^{3,1}$ from
which we can generate any other vector in the orbit. In particular, we
define the map $L_k(p)\in\ms L$ that takes $k$ to $p$, or briefly: $p
= L(p)k$. These $k$'s will be our building blocks for our
representation.
%
How do we recover the basis vectors $\ket{p\sg}$ from $\ket{k\sg}$?
Define the state $\ket{p\sg}$ by: \[ \ket{p\sg} \coloneqq N(p)U(L(p))
\ket{k\sg},\] where $N(p)$ is a phase factor that is important, but
will be determined later.

\textbf{Part Two: Induction}
Now we make a little jump, to show that if we have a representation for the
subspace spanned by $\{\ket{k\sg}\}_{\sg}$ then we can \emph{induce} a
representation for the subspace corresponding to the orbit of $k$:
$\{\ket{p\sg} \;|\; p\in(\ms L\cdot k), \sg \}$.
\begin{align*}
    U(\Lambda)\ket{p\sg}
&= N(p)U(\Lambda)U(L_p)\ket{k\sg}\\
&= N(p)U(L_{\Lambda p})U(L^{-1}_{\Lambda p})U(\Lambda)U(L_p)\ket{k\sg}\\
&= N(p)U(L_{\Lambda p})U\biggl(\underbrace{L^{-1}_{\Lambda p}\cdot\Lambda\cdot
L_p}_{W(\Lambda,p)}\biggr)\ket{k\sg}\\
&= N(p) U(L_{\Lambda p}) \sum_{\sg'} D_{\sg'\sg}(W(\Lambda,p)) \ket{k\sg'}\\
&= N(p) \sum_{\sg'} D_{\sg'\sg}(W(\Lambda,p)) \FR{\ket{\Lambda
p,\sg'}}{N(\Lambda p)}\\
&= \PFR{N(p)}{N(\Lambda p)} \sum_{\sg'} D_{\sg'\sg}(W(\Lambda,p))
\ket{\Lambda p,\sg'}
\end{align*}
Here is a quick explanation of the calculation. The second equality is
just inserting an identity. The third equality combines the Lorentz
transformations so that $W(\Lambda,p)$ is a Lorentz transformation
that fixes $k$ (easy exercise). The fourth line is the application of
the representation on the $\{\ket{k\sg}\}_{\sg}$ space: \[U(W)
\ket{k\sg} = \sum_{\sg'} D_{\sg'\sg}(W)\ket{k\sg}.\] The fifth and
sixth equalities are rearrangements and simplifications. To be
precise, the representation that we use in the fourth line is a
representation of the stabiliser group of $k$: \[\Stab(k)=\{W\in\ms
L\;|\;Wk=k\}\seq\ms L.\] In physics, we call this the ``little
group''. Notice that there is a different little group depending on
which representative $k$ we are dealing with.

\subsubsection{Representations of the Massive Little Group}
There are three orbits of interest and thus three possible little
groups: massive $k=(0,0,0,M)$, massless $k=(0,0,\kappa,\kappa)$,
vacuum. The vacuum case is straightforward so let's consider the other
two cases.

The little group is the stabiliser of $k$. In the massive case, these
are by inspection $\SO(3)\seq \ms L$ the rotation matrices sitting
inside the homogeneous orthochronous Lorentz group. The
representations of $\SO(3)$ can be labelled by $j=0, \FR{1}{2}, 1,
\dots$; let's denote them as $D^{(j)}_{\sg'\sg}(R)$ where $R \in
\SO(3)$. The induced representation is given by:
\[ U(\Lambda)\ket{p\sg} = \FR{N(p)}{N(\Lambda p)} \sum_{\sg'}
D^{(j)}_{\sg'\sg}(W_{\Lambda,p})\ket{\Lambda p,\sg'}\]
where $W(\Lambda,p) = L^{-1}_{\Lambda p}\cdot\Lambda \cdot L_p$ can be
computed explicitly for any $p$ in the same orbit as $k = (0,0,0,M)$.
Let's not devote the next two sections to the massless case.

\subsubsection{The Massless Little Group: Warm-Up}
The little group for the massless vector $(0,0,\kappa,\kappa)$ is
isomorphic to $ISO(2)\cong \bR^2\rtimes \SO(2)$. The proof of this is
a bit more involved, but since the result is so interesting let's
reproduce it.

\begin{prop} 
    Let $k = (0,0,1,1)\in\bR^{3,1}$. If we denote the stabiliser of
    $k$ by $\Stab(k)\seq\ms L$, then $\Stab(k)\cong ISO(2)$.
\end{prop}
\begin{proof}(Weinberg)
    Fix $W\in \Stab(k)$.  Choose $t = (0,0,0,1)$ to be a time-like
    vector. Then $Wt$ has the following two properties:
    \begin{align*}
        (Wt)^2 &= t^2 = -1 &
        (Wt)\cdot k &= t\cdot k = -1.
    \end{align*}
    Since both $k$ and $t$ have simple forms, it is straightforward to
    check that $Wt$ has the form:
    \[ Wt = (\alpha,\beta,\zeta,1+\zeta),\ \ \text{with}\ \
    \zeta=(\alpha^2+\beta^2)/2\]
    In particular, $W$ acting on $t$ is equal to the action of a
    simple Lorentz transformation $S(\alpha,\beta)$:
    \[ Wt = S(\alpha,\beta) t = 
        \bP
        1 & 0     & -\alpha & \alpha \\
        0 & 1     & -\beta  & \beta  \\
   \alpha & \beta & 1-\zeta & \zeta  \\
   \alpha & \beta & -\zeta  & 1+\zeta
        \eP
    \bP0\\0\\0\\1\eP\]
    Finally, this means that $S(\alpha,\beta)^{-1}W = R(\theta)$ is a
    rotation keeping the $t$ axis invariant. Rearranging we have the
    decomposition:
    \[W(\theta,\alpha,\beta)=S_{\alpha,\beta}R_\theta.\]
    By brute force multiplication, we can look for Abelian subgroups,
    $S_{\alpha,\beta}S_{\alpha',\beta'} = S_{\alpha+\alpha',\beta+
    \beta'}$ and $R_\theta R_{\theta'}=R(\theta+\theta')$
    and also for an isomorphism with $ISO(2)$: $R_\theta S_{\alpha,
    \beta} R^{-1}_\theta = S(\alpha\cos\theta +\beta\sin\theta,
    -\alpha\sin\theta+\beta\cos\theta)$.  The last equality shows what
    an observer $\ms O'$ (who is rotated by an angle $\theta$ from
    $\ms O$) would measure if $\ms O$ moved an object by a vector
    $(\alpha,\beta)$.
\end{proof}

\begin{remark}
Since $ISO(2)$ contains Abelian subgroups, $ISO(2)$ is not semi-simple.
\end{remark}

\subsubsection{The Massless Little Group}
Let's compute $W(\Lambda,p)$. Infinitesimally $W(\theta,\alpha,\beta)
= \iden + \om$ where 
\[ \om = 
    \left( \begin{smallmatrix}
        0 &  \theta & -\alpha & \alpha \\
  -\theta &    0    & -\beta  & \beta  \\
   \alpha &  \beta  &  0 & 0\\
  -\alpha & -\beta  &  0 & 0
   \end{smallmatrix} \right)
   \implies 
   U(W_{\theta,\alpha,\beta}) = \iden + i\alpha A + i\beta B + i\theta
   J_3
\]
where $A = J_2+K_1$, and $B = -J_1+K_2$. The algebra leads to a
surprise. The commutator $[A,B]=0$, and $[H,A]=[H,B]=0$ which means
that we can introduce new quantum numbers: $a,b$ corresponding to
$A,B$. However by applying a rotation by $\theta$ we get new distinct
states which are also eigenstates labelled by the same $k$-vector.

Input from experimental observations: massless particles are not
observed to have a continuous symmetry. We must conclude that
$\ket{k\sg}$ must vanish under $A$ and $B$. The only remaining
operator for which we may distinguish $\ket{k\sg}$ is $J_3$ and so we
write: \[ J_3\ket{k\sg} = \sg\ket{k\sg}. \]
The value $\sg$ is the magnitude of the angular momentum in the
direction of $k=(0,0,1,1)$, or more precisely in the $\bm {\hat k}$
direction. This is called the \textbf{helicity} of the massless
particle.

\subsubsection{Representations of the Massless Little Group}
We know, from the Abelian subgroup structure, $U(S_{\alpha,\beta})
= e^{i\alpha A + i\beta B}$ and $U(R_\theta) = e^{i\theta J_3}$.
Therefore for any $W=S_{\alpha,\beta}R_\theta$ we may use the
definition of the representation $U(W) = U(S_{\alpha,\beta})
U(R_\theta)$ so that by acting on $\ket{k\sg}$ we get: $U(W)
\ket{k\sg}=e^{i\theta\sg}\ket{k\sg}$. Remember: $A\ket{k\sg} =
B\ket{k\sg}=0$. Therefore:
\begin{align*}
    U(W)\ket{k\sg} &= \sum_{\sg'} D_{\sg'\sg}(W)\ket{k\sg},  &
    D_{\sg'\sg}(W) &= e^{i\theta\sg}\delta_{\sg'\sg}
\end{align*}
Inducing this action to the subspace corresponding to the entire
orbit, $\ms L \cdot k$, we get:
\begin{align*}
    U(\Lambda)\ket{p\sg} &= \FR{N(p)}{N(\Lambda p)} %\sum_{\sg'}
    e^{i\sg\theta(\Lambda,p)}\ket{\Lambda p,\sg} &
    W(\Lambda,p)&\equiv L^{-1}_{\Lambda p}\Lambda L_p \equiv
    S_{\alpha(\Lambda,p),\beta(\Lambda,p)}R_{\theta(\Lambda,p)}
\end{align*}

\begin{remark} The part of the little group parameterized by
$\alpha,\beta$ will be responsible for electromagnetic gauge
invariance. Stay tuned, don't change the channel!
\end{remark}

\begin{remark} As we saw in our discussion of projective
representations, the fact that the Lorentz group has $\pi_1(\ms L)
= \bZ_2$ imposes the constraint that
$\sg\in\{\dots,-\FR{1}{2},0,\FR{1}{2},1,\dots\}$.  The statment is
that $\iden\to\Lambda=e^{i(2\pi)J_3} \to\iden$ picks up a phase
$e^{2\pi i\sg}$ for a state $\ket{k\sg}$. Doing this twice would pick
up $e^{4\pi i \sg}$. However, the path $\iden\to\Lambda\to \iden \to
\Lambda \to\iden$ is contractible to a point, which means that
$e^{4\pi i\sg}=1$ and so $4\pi\sg\equiv 0 \pmod{2\pi}$ or that $\sg$
is an integer or half-integer.  
\end{remark}

\begin{remark}
    Spatial inversion, as we shall see, connects states of helicity
    that differ by a sign $(-1)$. Electromagnetism and gravitation
    both are spatially symmetric. Electroweak theory is not spatially
    symmetric (cf. nuclear beta decay).
    \begin{align*}
        \text{ photons}&: \sg = \pm1 &
      \text{ neutrinos}&: \sg = +1/2\\
      \text{ gravitons}&: \sg = \pm2&
  \text{ antineutrinos}&: \sg = -1/2
    \end{align*}
\end{remark}

\subsubsection{Polarization}

\begin{prop} Helicity is Lorentz invariant.
\end{prop}
\begin{proof} This follows from the induced action:
$U(\Lambda)\ket{p\sg} = \FR{N(p)}{N(\Lambda p)} 
e^{i\sg\theta(\Lambda,p)}\ket{\Lambda p,\sg}$.
\end{proof}

\newcommand{\ap}{\alpha_{\scriptscriptstyle +}}
\newcommand{\an}{\alpha_{\scriptscriptstyle -}}
\newcommand{\apm}{\alpha_{\scriptscriptstyle \pm}}
The above proposition does not mean that the state is the same.
A general one-photon state is a superposition:
\[ \ket{p;\alpha} = \ap \ket{p;+1} + \an\ket{p;-1} \]
We say the photon is
\begin{itemize}
    \vspace{-.1cm}
\item \textbf{elliptically polarized} if $|\apm|\ne0$ and $|\ap| \ne |\an|$, 
    \vspace{-.3cm}
\item \textbf{circularly polarized} if only one of $\ap=0$ or $\an=0$,
    \vspace{-.3cm}
\item \textbf{linearly polarized} if $|\ap| = |\an|$.
\end{itemize}
The relative phase may change as we boost in certain directions.

\subsection{Space Inversion and Time Reversal}

\begin{mdframed}[%
linewidth=2,roundcorner=10pt,leftmargin=20,
rightmargin=20,backgroundcolor= yellow!40,
linecolor=blue!70!black,frametitle=The Fight for a True Representation] 
Can we enforce $U(\Lambda,a)U(\Lambda',a') = U(\Lambda\Lambda',\Lambda
a'+a)$ for the entire Poincar\'e group, including the cases when
parity and time reversal $\ms P$ and $\ms T$ are allowed?
\end{mdframed}

Suppose there are operators $\rP = U(\ms P,0), \rT = U(\ms T,0)$ 
satisfying 
\begin{align}
\rP U(\Lambda,a)\rP^{-1}&=U(\ms P\Lambda\ms P^{-1},\ms Pa)\label{eqn:parity}\\ 
\rT U(\Lambda,a)\rT^{-1}&=U(\ms T\Lambda\ms T^{-1},\ms Ta)\label{eqn:timrev}
\end{align}
Nuclear beta decay shows that the first of these is only approximate
and the second seems also approximate (PRL 13 138, 1964). Still,
suppose $\rP$ and $\rT$ do exist satisfying \eqref{eqn:parity} and
\eqref{eqn:timrev}. Applying \eqref{eqn:parity}, \eqref{eqn:timrev} to
the infinitesimal operators of the Poincar\'e group we get:
\begin{align*}
    \rP iH \rP^{-1} &= iH, &
    \rT iH \rT^{-1} &=-iH.
\end{align*}
From these two equations it immediately follows that $\rP$ and $\rT$
must be respectively unitary and linear, and antiunitary and
antilinear. 

How do $\rP$ and $\rT$ transform one-particle states?
It is useful to collect the commutation relations:
\begin{align}
    \rP \bm J \rP^{-1} &= +\bm J\label{eqn:PJ}\\
    \rP \bm K \rP^{-1} &= -\bm K\label{eqn:PK}\\
    \rP \bm P \rP^{-1} &= -\bm P\label{eqn:PP}\\
    \rP  H \rP^{-1} &=  H\label{eqn:PH}\\
    \rT \bm J \rT^{-1} &= -\bm J\label{eqn:TJ}\\
    \rT \bm K \rT^{-1} &= +\bm K\label{eqn:TK}\\
    \rT \bm P \rT^{-1} &= -\bm P\label{eqn:TP}\\
    \rT  H \rT^{-1} &=  H\label{eqn:TH}
\end{align}

\textbf{Parity, $\bm{M>0}$.}
$\ket{k\sg}$ are eigenvectors of $\bm P, H, J_3$ with eigenvalues
$(0,M,\sg)$. Applying $\rP$: takes $0\to -0$, $M\to M$, $\sg\to\sg$;
thus $\rP\ket{k\sg} = \eta_\sg\ket{k\sg}$ where $\eta_\sg\in U(1)$ is
a phase. By applying $J_\pm = J_1\pm iJ_2$, we can show that $\eta$ is
independent of $\sg$. For finite spatial momentum states,
using $\ms P L(p) \ms P^{-1} = L(\ms P p)$ we get:
\[\rP \ket{p,\sg} = \eta\ket{\ms Pp,\sg}.\]

\textbf{Time Reversal, $\bm{M>0}$.}
$\ket{k\sg}$ are eigenvectors of $\bm P, H, J_3$ with eigenvalues
$(0,M,\sg)$. Applying $\rT$: takes $0\to -0$, $M\to M$, $\sg\to-\sg$;
thus $\rT\ket{k,\sg} = \zeta_\sg\ket{k,-\sg}$ where $\zeta_\sg$ is
a phase. For finite spatial momentum states,
\[\rT \ket{p,\sg} = \zeta(-1)^{j-\sg}\ket{\ms Pp,-\sg}.\]


\textbf{Parity, $\bm{M=0}$.}
Consider $\ket{k\sg}$ with eigenvalue $k=(0,0,\kappa,\kappa)$ under $P^\mu$ and eigenvalue $\sg$ under $J_3$.
Applying $\rP$, yields $k\to (\ms Pk) = (0,0,-\kappa,\kappa)$, and eigenvalue $\sg$ under $J_3$.
The helicity $\bm J\cdot \hat k$ changes $-\sg$ (since now it is antiparallel, if it originally was parallel).
After some analysis we get:
\[ \rP \ket{p,\sg} = \eta_\sg e^{\mp i\pi\sg}\ket{\ms Pp,\sg},\]
check Weinberg for the details in the sign. They intrinsically come
from the fact that $\pi_1(\text{Poincar\'e})=\bZ_2$.

\textbf{Time Reversal, $\bm{M=0}$.}
Consider $\ket{k\sg}$ with eigenvalue $k=(0,0,\kappa,\kappa)$ under $P^\mu$ and eigenvalue $\sg$ under $J_3$.
Applying $\rT$, yields $k\to (\ms Pk) = (0,0,-\kappa,\kappa)$, and eigenvalue $-\sg$ under $J_3$.
The helicity $\bm J\cdot \hat k$ does not change $\sg$.
After some analysis we get: $\rT \ket{p,\sg} = \zeta_\sg e^{\pm i\pi\sg}\ket{\ms Pp,\sg}$.

\subsubsection{$\rT^2$} 
Working the action of $\rT^2$ on $\ket{p\sg}$ yields:
\[ \rT^2\ket{p\sg} = e^{\mp 2i\pi\sg}\ket{p\sg} \equiv (-1)^{2|\sg|}\ket{p\sg} \]
If a state contains an odd-number of particles with half-integer spin
there is an overall change of sign: $\rT^2\Psi = -\Psi$. This is
independent of the direction of spatial momentum and so even in the
presence of interactions which only obey time-reversal this result is
preserved.  

If $\Psi$ is an eigenstate, then $\rT\Psi$ is also an eigenstate if
$[H,\rT]=0$. Moreover, $\rT\Psi\not\propto\Psi$ otherwise:
$-\Psi=\rT^2\Psi=\rT(\alpha\Psi) = \alpha^*\rT\Psi =
|\alpha|^2\Psi$, which is a contradiction. This is called a
\textbf{Kramers degeneracy}.
%
If the system had rotational symmetry, then an odd-number of
half-integer states means a total spin of $2j+1=2,4,6,\dots$, and that
is an even degeneracy. Kramers result is that even when rotational
invariance is broken, say by electric fields, there is still a
two-fold degeneracy as long as the Hamiltonian is invariant under
$\rT$.

This can rule out electric dipole moments: suppose there is an
anomalous electric dipole moment of a particle, then in the presence
of a static electric field, the particle would choose a preferred
direction thus breaking the spin degeneracy -- contradicting the
Kramers degeneracy, or more precisely, time-reversal invariance.



\subsection{Appendix: Projective Representations and Central Extensions}
%\begin{comment}
In this subsection, we shall understand where the terms {2-cocycle}
and {central charge} come from. Skipping this section will not 

\begin{center}
    \begin{tikzpicture}
        \draw (0,0) node {Projective Rep};
        \draw (2.6,0) node {Phase $\phi$};
        \draw (5.5,0) node {$[\phi]\in H^2(G,U(1))$};
        \draw (8.5,-0.037) node {$U(1)\rtimes_{\tilde\phi} G$};
        \draw[->] (1.2,0) -- +(.7,0);
        \draw[->] (3.3,0) -- +(.7,0);
        \draw[->] (6.9,0) -- +(.7,0);
    \end{tikzpicture}
\end{center}

Let us quickly introduce/review the gadgets from group cohomology that
will be useful to us. We shall start from cochains.
\begin{defn}
    Let $G$ be a group, and $A$ be an Abelian group, which admits a
    right $G$-action. An \mbox{\textbf{n-cochain}} is a function 
    \mbox{$f:\overbrace{G\times\cdots\times G}^n\to A$}. 
    %
    Let $C^n(G,A)$ refer to the set of all $n$-cochains, and also
    let's write $C^0(G,A) = A$.
    %
    Define a map $d:C^n(G,A)\to C^{n+1}(G,A)$ that takes an $n$-chain
    $f$ and returns an $n+1$ chain $df$ defined by the formula:
    \[ df(g_1,\dots,g_{n+1})
        = f(g_2,\dots,g_{n+1}) 
        \left(
        \prod_{i=1}^{n} f(g_1,..,g_{i-1},g_ig_{i+1},g_{i+2},..,g_{n+1})^{(-1)^i}
        \right)
        (f(g_1,\dots,g_{n})^{(-1)^{n+1}}\cdot g_{n+1})
    \]
    Denote by 
    \vspace{-.5cm}
    \begin{align*}
    Z^n(G,A) &= \ker(d:C^n\to C^{n+1}),         \quad\!\text{the \textbf{n-cocycles},}\\
    B^n(G,A) &= \mathrm{im}(d:C^{n-1}\to C^n),  \quad\text{the \textbf{n-coboundaries},}\\
    H^n(G,A) &= Z^n(G,A)/B^n(G,A), \quad\text{the \textbf{n$^\text{th}$-cohomology group}.}
    \end{align*}
    Note that $C^n,Z^n,B^n$, and $H^n$ are all groups with the
    addition endowed from $C^n$.
\end{defn}
\begin{example}
If the group action is trivial, that is $x\cdot g = x$ for any
$g\in G$ and $x\in A$, then 
\begin{align*}
Z^2(G,A) 
&= \{f(x,y)| f_{y,z} f_{xy,z}^{-1} f_{x,yz} f_{x,y}^{-1} = 1\}\\
&= \{f(x,y)| f_{y,z}f_{x,yz}  = f_{x,y}f_{xy,z}\}
\end{align*}
Compare this with the condition (equation~\ref{eqn:phasecocycle1})
on the phase $\phi$: $\phi(T_1,T_2T_3) + \phi(T_2,T_3) \equiv
\phi(T_1T_2,T_3)+\phi(T_1,T_2) \pmod{2\pi}$.  The group $A$ in this
example is $\bR/2\pi\bZ$ with additive operation. This is where the
term 2-cocycle comes from. What are the 2-coboundaries? For arbitrary
$G$ and $A$:
\begin{align*}
    B^2(G,A) &= \{f_\lam(x,y) | f(x,y) = \lam(y)\lam(xy)^{-1}\lam(x)^y\}
\end{align*}
Comparing this with the ``trivial phases'' in
example~\ref{example:trivialphase1}, we see these trivial phases are
precisely the 2-coboundaries. The 2$^\text{nd}$ cohomology group then
the equivalence class of phases modulo the addition of ``trivial
phases'' which is precisely what we have been studying above.
Finally, since $\bR/2\pi\bZ\cong U(1)$ as Abelian groups, it makes no
difference which Abelian group $A$ we pick when speaking of
$H^2(G,A)$. \footnote{Our $G$ should be one of the Lorentz group,
projective Lorentz group, or the image of the Lorentz group under the
representation.}
\end{example}

Now that we've seen that each projective representation gives rise to
a cohomology class, how do these relate to Abelian central extensions?

\begin{defn}
    Let $G$ and $A$ be groups, with $A$ Abelian. We say that $E$ is an
    \textbf{group extension of $G$ by $A$} if there is a short exact sequence
    \[ \begin{tikzcd} 1\arrow[r] &A\arrow[r,"i"] &E\arrow[r,"\pi"] &G\arrow[r] &1
    \end{tikzcd}\!.\]
\end{defn}
\begin{prop} If $1\to A\to E\to G\to 1$ is a group extension with a
    set-theoretic section $s:G\to E$ satisfying $\pi\circ s =
    \iden_G$, then there is an action of both $E$ and $G$ on $A$ by
    conjugation. Moreover $A$ is normal in $E$.  
\end{prop}
Morally speaking, we should think of $A$ and $G$ sitting inside of
$E$. We may abuse notation $i(a) \equiv a$ and $s(g) \equiv g$, but
not forgetting that $\pi(a) = 0$ and $\pi(g) = g$.  The $E$-action and
$G$-action are obvious: $e\cdot a = eae^{-1}$ and $g\cdot a=gag^{-1}$. 
It is easy with this notation to check that these actions are
well-defined. Here is the formal proof.
\begin{proof}
    Let $a\in A, e\in E, g\in G$. Define $e\cdot a \in A$ be the
    unique element such that $i(e\cdot a) = e\,i(a)\,e^{-1}$.  We may
    check that $e\,i(a)\,e^{-1}$ is indeed in $i(A)$, because
    $\pi(e\,i(a)\,e^{-1}) = \pi(e)[\pi\circ i(a)]\pi(e)^{-1} = 1$ and
    exactness at $E$ shows that $e\,i(a)\,e^{-1}\in \ker\pi = i(A)$.
    We similarly get a $G$-action on $A$, by writing $g\cdot a =
    s(g)i(a)s(g)^{-1}$.  
\end{proof}
%%%%%%%%%%%%%% MAKE THIS MORE GENERAL: use less sections.
In fact, we do not need sections for this proof to go through. Exercise.
%%%%%%%%%%%%%% MAKE THIS MORE GENERAL: use less sections.

\begin{defn}
    Let $1\to A\to E\to G\to 1$ be an extension of groups. We say
    $E$ is a \textbf{central extension} if the image of $A$ lies in
    the centre of $E$. In particular, this means that the $G$-action
    on $A$ is trivial, because $G$ acts by conjugation and $A$
    commutes with every element.
\end{defn}

\begin{center} $\bm{ E \cong A\rtimes_f G }$ \end{center}

Let us assume that $1\to A\to E \to G\to 1$ is a central extension.
%
Right now, we do not know what the structure of $E$ is except that
it's an extension.  Our next goal is to show that $E$ can be
characterized by a single function $f:G\times G\to A$. 
%
Let us choose a set-theoretic section $s:G\to A$, which means that in
general we have $s(g)s(h)\ne s(gh)$.  To measure how poorly $s$ fails
to be a homomorphism we may introduce a function $f:G\times G\to A$
defined by: \[ s(g)s(h) = i(f_{g,h})s(gh). \]

We may now define the map from $\phi:A\times G \to E$ given by
$(a,g)\mapsto i(a)s(g)$. This is a bijection, as one can easily check.
Let us put the following group multiplication on $A\times G$:
\[(a_1,g_1)*(a_2,g_2) = (a_1(g_1\cdot a_2)f_{g_1,g_2}, g_1g_2).\] 
It is easy to check that this is well-defined and makes $A\times G$
into a group. In fact, it is chosen so that operation is exactly the
one needed to make $\phi:A\times G\to E$ an isomorphism:
\begin{align*}
\phi(a_1,g_1)\cdot \phi(a_2,g_2)
&= i(a_1)s(g_1)i(a_2)s(g_2)
= i(a_1(g_1\cdot a_2))s(g_1)s(g_2)
= i(a_1(g_1\cdot a_2))i(f_{g_1,g_2})\,s(g_1g_2)\\
&= (\, a_1(g_1\cdot a_2)f_{g_1,g_2},\,s(g_1g_2)\,)
\end{align*}
Associativity of this multiplication imposes the following constraint
on the function $f$. Remember, since $E$ is a central extension, that
the $G$-action on $A$ is trivial:
%   \begin{align*}
%       [(a_1,g_1)(a_2,g_2)](a_3,g_3) &= (a_1,g_1)[(a_2,g_2)(a_3,g_3)]\\
%       (a_1(g\cdot a_2)f_{g_1,g_2},g_1g_2)(a_3,g_3) &= (a_1,g_1)(a_2(g_2\cdot a_3)f_{g_2,g_3},g_2g_3)\\
%       (a_1(g\cdot a_2)f_{g_1,g_2}(g_1g_2\cdot a_3)f_{g_1g_2,g_3},g_1g_2g_3) 
%       &= (a_1(g_1\cdot [a_2(g_2\cdot a_3)f_{g_2,g_3}])f_{g_1,a_2(g_2\cdot a_3)f_{g_2,g_3}},g_1g_2g_3)
%   \end{align*}
%   This is a horrible mess. However, if $A$ is in the centre of $E$,
%   which in particular implies that the $G$-action is trivial, the
%   associativity conditions simplifies to:
\vspace{-0.2cm}
\begin{align*}
    [(a_1,g_1)(a_2,g_2)](a_3,g_3) &= (a_1,g_1)[(a_2,g_2)(a_3,g_3)]\\
    (a_1a_2f_{g_1,g_2},g_1g_2)(a_3,g_3) &= (a_1,g_1)(a_2a_3f_{g_2,g_3},g_2g_3)\\
    (a_1a_2a_3f_{g_1,g_2}f_{g_1g_2,g_3},g_1g_2g_3) &= (a_1a_2a_3f_{g_2,g_3}f_{g_1,g_2g_3},g_1g_2g_3)\\
    \implies\qquad\qquad f(g_1,g_2)f(g_1g_2,g_3) &= f(g_2,g_3)f(g_1,g_2g_3),
\end{align*}
which says that $f$ is a $2$-cocycle, or in other words $f$ defines an
equivalence class $[f]\in H^2(G,A)$.


%\end{comment}

\begin{comment}
\subsubsection{Exit: Mathematics.}
What have we shown with our mathematical interlude?

\[
\begin{tikzcd}
    \text{Projective representation}\arrow[r]
    & \text{Phase $\phi$}\arrow[r] 
    & \text{$[\phi] \in
    H^2(G,\bR/2\pi\bZ)$}\arrow[r,semithick,blue,dashed]
    & \colb{ U(1)\rtimes_{\phi} G}
\end{tikzcd}
\]

For every projective representation, we have shown that we may
associate a cohomology class $[\phi]$. If this class is cohomologous
to $[0]$, then we may redefine this projective representation to get a
linear representation. If not, then $[\phi]$ determines uniquely a
central extension $U(1)\rtimes_\phi G$. It is a fact that we may lift
the projective representation from $G\to U(V)$ to a linear
representation on $U(1)\rtimes_\phi G$. 

The question whether or not there exist \emph{any} intrinsically
projective representations of $G$ reduces to the question of computing
$H^2(G,A)$. In our case, $G$ is the inhomogeneous orthochronous
Lorentz group whose complexification is $\SL_2(\bC)/\bZ_2$. 
%
The computation of $H^2(\text{inhom. ortho. Lorentz},U(1)\,)$ can be done\dots
by looking at $\pi_1( \SL_2(\bC)/\bZ_2 )$ !?
%

\end{comment}



%\input{FreeFields}
%\input{Scattering}
%\input{Radiative-corrections}





\end{document}
